\section{Tests}
I test sono contenuti nel file tests.py
Può essere eseguito con il comando ``nosetests'' oppure ``python tests.py''
\subsection{Test sulle funzionalità}
\paragraph{\textbf{Connessione}}:
\begin{lstlisting}[language=python]
def test0_connection(self):
\end{lstlisting}
Questo test verifica che il server sia raggiungibile e accetti le connessioni.
\paragraph{\textbf{Login}}:
\begin{lstlisting}[language=python]
def test10_empty_login(self):
\end{lstlisting}
Questo test verifica che un login con user e password vuoti sia gestito correttamente dal server
\begin{lstlisting}[language=python]
def test11_login(self):
\end{lstlisting}
Questo test prova l'autenticazione con un utente e una password validi. Fallisce se il server ritorna un codice diverso da 1
\paragraph{\textbf{Registrazione}}:
\begin{lstlisting}[language=python]
def test3_wrong_register(self):
\end{lstlisting}
Questo test esegue prova a registrare una categoria merceologica già presente(``libri''). Da esito positivo se il server ritorna il codice 0

\begin{lstlisting}[language=python]
def test4_wrong_register2(self):
\end{lstlisting}
Questo test esegue prova a registrare una categoria vuota (``''). Da esito positivo se il server ritorna il codice 0

\begin{lstlisting}[language=python]
def test5_register(self):
\end{lstlisting}
Questo test registra una categoria merceologica non presente nel sistema (``elettrodomestici''). Ha successo se il server ritorna il codice 1

\begin{lstlisting}[language=python]
def test92_register_unlogged(self):
\end{lstlisting}
Questo test prova a registrare una categoria merceologica senza autenticazione. Ha successo se il server ritorna 3.

\paragraph{\textbf{Vendita}}:
\begin{lstlisting}[language=python]
def test6_sell(self):
\end{lstlisting}
Questo test inserisce un prodotto in vendita, si aspetta una risposta dal server uguale a 1.

\begin{lstlisting}[language=python]
def test7_wrong_sell(self):
\end{lstlisting}
Questo test inserisce un prodotto in vendita vuoto (``''). Si attete una risposta dal server uguale a 0.

\paragraph{\textbf{Offerta}}:

\begin{lstlisting}[language=python]
def test8_sell_bid(self):
\end{lstlisting}
Questo test inserisce un nuovo prodotto in vendita e prova a fare un offerta. Il risultato atteso è un fallimento in quanto non è possibile, da specifica, fare un offerta per il proprio prodotto. Risultato atteso 5

\begin{lstlisting}[language=python]
def test9_sell_wrongbid(self):
\end{lstlisting}
Questo test prova a inserire un offerta per un prodotto non esisteente. Il risultato atteso è 5.

\begin{lstlisting}[language=python]
def test93_close_auction_nowinner(self):
\end{lstlisting}
Questo test inserisce un prodotto e chiude l'asta. Il risultato atteso è 8, perchè non c'è nessun vincitore.


\begin{lstlisting}[language=python]
def test94_fail_close_auction(self):
\end{lstlisting}
Questo test effettua un offerta su un oggetto inserito da un'altro utente, dopodichè prova a chiudere l'asta.
Il risultato atteso è 7 in quanto l'utente non è autorizzato a chiudere l'asta


\subsection{Test sulle notifiche}
Quest verificano la funzionalità delle notifiche asincrone.

\begin{lstlisting}[language=python]
def test2_notify(self):
\end{lstlisting}
Quest test verfica che il server risponda correttamente quando si registra l'endpoint di notifica.

\begin{lstlisting}[language=python]
def test991_notif_win(self):
\end{lstlisting}
Questo test effettua la connessione e l'autenticazione con due utenti. Il primo utente crea un prodotto dopodichè il secondo utente effettua un offerta per il prodotto e il primo utente chiude l'asta.

Il test verifica che  il secondo utente riceva la notifica con il codice corrispondente alla vincita dell'asta.
\begin{lstlisting}[language=python]
def test992_notif_close_auct(self):
\end{lstlisting}
Questo test esegue la stessa procedura del test precedente,  ma verifica che il secondo client riceva il codice corrispondente alla chiusura dell'asta.


\begin{lstlisting}[language=python]
def test993_notif_manybids(self):
\end{lstlisting}
Questo test simula un'asta completa.
1 utente crea l'asta e altri 2 utenti effettuano varie offerte.
Il test fallisce se ogni utente riceve i messaggi di notifica corrispondenti al superamento della propria offerta da parte di un'altro utente.