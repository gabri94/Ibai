% !TEX TS-program = pdflatex
% !TEX encoding = UTF-8 Unicode

% This is a simple template for a LaTeX document using the "article" class.
% See "book", "report", "letter" for other types of document.

\documentclass[11pt]{article} % use larger type; default would be 10pt

\usepackage[utf8]{inputenc} % set input encoding (not needed with XeLaTeX)
\usepackage{listings}
\usepackage{hyperref}
\hypersetup{
    colorlinks,
    citecolor=black,
    filecolor=black,
    linkcolor=blue,
    urlcolor=black
    linktoc=section
}
%%% Examples of Article customizations
% These packages are optional, depending whether you want the features they provide.
% See the LaTeX Companion or other references for full information.

%%% PAGE DIMENSIONS
\usepackage{geometry} % to change the page dimensions		
\geometry{a4paper} % or letterpaper (US) or a5paper or....
% \geometry{margin=2in} % for example, change the margins to 2 inches all round
% \geometry{landscape} % set up the page for landscape
%   read geometry.pdf for detailed page layout information

\usepackage{graphicx} % support the \includegraphics command and options

% \usepackage[parfill]{parskip} % Activate to begin paragraphs with an empty line rather than an indent

%%% PACKAGES
\usepackage{booktabs} % for much better looking tables
\usepackage{array} % for better arrays (eg matrices) in maths
\usepackage{paralist} % very flexible & customisable lists (eg. enumerate/itemize, etc.)
\usepackage{verbatim} % adds environment for commenting out blocks of text & for better verbatim
\usepackage{subfig} % make it possible to include more than one captioned figure/table in a single float
% These packages are all incorporated in the memoir class to one degree or another...

%%% HEADERS & FOOTERS
\usepackage{fancyhdr} % This should be set AFTER setting up the page geometry
\pagestyle{fancy} % options: empty , plain , fancy
\renewcommand{\headrulewidth}{0pt} % customise the layout...
\lhead{}\chead{}\rhead{}
\lfoot{}\cfoot{\thepage}\rfoot{}

%%% SECTION TITLE APPEARANCE
\usepackage{sectsty}
\allsectionsfont{\sffamily\mdseries\upshape} % (See the fntguide.pdf for font help)
% (This matches ConTeXt defaults)

%%% ToC (table of contents) APPEARANCE
\usepackage[nottoc,notlof,notlot]{tocbibind} % Put the bibliography in the ToC
\usepackage[titles,subfigure]{tocloft} % Alter the style of the Table of Contents
\renewcommand{\cftsecfont}{\rmfamily\mdseries\upshape}
\renewcommand{\cftsecpagefont}{\rmfamily\mdseries\upshape} % No bold!

%%% END Article customizations

\usepackage{bera}% optional: just to have a nice mono-spaced font
\usepackage{listings}
\usepackage{xcolor}

\colorlet{punct}{red!60!black}
\definecolor{background}{HTML}{EEEEEE}
\definecolor{delim}{RGB}{20,105,176}
\colorlet{numb}{magenta!60!black}


\lstdefinelanguage{json}{
    basicstyle=\normalfont\ttfamily,
    numbers=left,
    numberstyle=\scriptsize,
    stepnumber=1,
    numbersep=8pt,
    showstringspaces=false,
    breaklines=true,
    frame=lines,
    backgroundcolor=\color{background},
    literate=
     *{0}{{{\color{numb}0}}}{1}
      {1}{{{\color{numb}1}}}{1}
      {2}{{{\color{numb}2}}}{1}
      {3}{{{\color{numb}3}}}{1}
      {4}{{{\color{numb}4}}}{1}
      {5}{{{\color{numb}5}}}{1}
      {6}{{{\color{numb}6}}}{1}
      {7}{{{\color{numb}7}}}{1}
      {8}{{{\color{numb}8}}}{1}
      {9}{{{\color{numb}9}}}{1}
      {:}{{{\color{punct}{:}}}}{1}
      {,}{{{\color{punct}{,}}}}{1}
      {\{}{{{\color{delim}{\{}}}}{1}
      {\}}{{{\color{delim}{\}}}}}{1}
      {[}{{{\color{delim}{[}}}}{1}
      {]}{{{\color{delim}{]}}}}{1},
}




%%% The "real" document content comes below...

\title{Implementazione di un sistema di Aste}
\author{Gabriele Gemmi\\gabriele.gemmi@stud.unifi.it}
\date{Febbraio 2017} % Activate to display a given date or no date (if empty),
         % otherwise the current date is printed

\begin{document}
\maketitle
\section*{Introduzione}
\tableofcontents
\newpage
\section{Protocollo}
Per sviluppare questo progetto era necessario definire un protocollo per la comunicazione client-server.
Per garantire  la scalabilità del sistema è stato scelto un protocollo stateless, nonchè connectionless.\\
E' stato utlizzato JSON come formato per lo scambio di dati in quanto erano disponibili nella libreria standard python i metodi per serializzare e deserializzare gli oggetti.\\

Il formato di un generico  messaggio è:
\begin{lstlisting}[language=json]
{
  "msg_id": <id>
}
\end{lstlisting}
La prima cifra del campo msg\_id indica il tipo di richiesta, la seconda indica il metodo
\begin{itemize}
\item -1: Messaggio di risposta a un comando
\item 1: Autenticazione
\begin{itemize}
\item 11: Log in
\item 12: Log out
\end{itemize}
\item 2: Notifiche
\begin{itemize}
\item 21: Registra l'endpoint per le notifiche
\item 22: Messaggio di notifica
\end{itemize}
\item 3: Categorie
\begin{itemize}
\item 31: Registra nuova categoria
\item 32: Ricerca di una categoria
\item 33: Lista tutte le categorie
\end{itemize}
\item 4: Prodotti
\begin{itemize}
\item 41: Registra un nuovo prodotto
\item 42: Ricerca un prodotto
\item 43: Lista tutti i prodotti
\end{itemize}
\item 5: Asta
\begin{itemize}
\item 51: Fai un offerta
\item 52: Cancella la tua ultima offerta
\item 53: Chiudi l'asta

\end{itemize}

\end{itemize}

\subsection{Messaggio di risposta}
Il formato di un generico messaggio di risposta è il seguente.\\
Se non è esplicitamente indicato un'altro formato per la risposta,
 allora la risposta sarà di questo formato.
\begin{lstlisting}[language=json]
{
  "msg_id": -1
  "response" : <code>
}
\end{lstlisting}

\begin{tabular}{|l | l | l |}
\hline
Campo & Tipo & Descrizione \\ \hline
response & int & '1' in caso di successo  \\
& & '-1' comando non sia valido \\
& & '0' errore \\
& & '2' sessione scaduta \\
& & '3' sessione non valida \\ \hline
\end{tabular}\\
Questo messaggio può essere esteso dalle varie funzioni aggiungendo campi e codici di risposta.\\

\subsection{Autenticazione}
La prima cosa che deve fare un client quando si connette al server è autenticarsi utilizzando un username ed una password.
Il server mantiene una lista di utenti con relative password, quando il client prova ad autenticarsi vengono confrontati e se sono corretti l'autenticazione è avvenuta.
Per evitare che nel server vengano memorizzate le password in chiaro, è stato scelto di memorizzare e trasmettere l'hash md5 della password.
Il Client si occuperà di eseguire l'hash md5 della password inserita dall'utente e di inviarlo al server, che lo confronterà con quello memorizzato.\\

\subsubsection{Sessioni}
Per gestire le sessioni  mantenendo un procollo stateless è stato scelto di generare un token ed inviarlo al client.
Una volta effettuato il login il server restituirà un token al client che verrà utilizzato per autenticare tutti i comandi.\\
Il token inviato al client è della forma:
\begin{lstlisting}[language=json]
{
  "username" : <user>
  "expiration": <expiration>
  "signature": <signed proof of autenticity>
}
\end{lstlisting}

\begin{tabular}{|l | l | l|}
\hline
Campo & Tipo & Descrizione \\ \hline
username & stringa & Username dell'utente \\ \hline
expiration & int & Data e ora di scadenza della sessione (in secondi) \\ \hline
signature & stringa & Codice di verifica del messaggio\\ \hline
\end{tabular}\\

Il campo signature è un hash dei precedenti due campi (username e expiration) più una chiave segreta generata dal server.
In questa maniera il server potrà verificare che il token sia autentico senza doverne mantenere una copia in memoria.
\subsubsection{Formato}
Il messaggio di autenticazione è del formato:
\begin{lstlisting}[language=json]
{
  "msg_id": 11,
  "user": <username>,
  "pass": <hash of pwd>
}
\end{lstlisting}

\begin{tabular}{|l | l | l |}
\hline
Campo & Tipo & Descrizione \\ \hline
user & stringa & Username dell'utente \\ \hline
pass & stringa & Hash MD5 della password \\ \hline
\end{tabular} \\
\vspace*{1em}

La risposta è del formato:
\begin{lstlisting}[language=json]
{
  "msg_id": -1,
  "response": <code>
  "token": <token>
}
\end{lstlisting}

\begin{tabular}{|l | l | l |}
\hline
Campo & Tipo & Descrizione \\ \hline
response & int & codice di risposta  \\ \hline
token & json & token per la sessione \\ \hline
\end{tabular} \\
\subsection{Gestione delle Notifiche}
Il server di aste può inviare delle notifiche asincrone ai client per comunicare aggiornamenti di stato.
Per esempio il client riceverà una notifica quando un'asta viene terminata o la sua offerta viene superata.\\
\subsubsection{Registrazione}
Per ricevere delle notifiche asincrone il client dovrà stare in ascolto su una determinata porta, e dovrà comunicare il proprio indirizzo e la porta al server. Per fare questo è disponibile un comando. Di seguito il formato:\\
\begin{lstlisting}[language=json]
{
  "token": <token>,
  "msg_id": 21,
  "host": <hostname>,
  "port": <port>
}
\end{lstlisting}

\begin{tabular}{|l | l | l |}
\hline
Campo & Tipo & Descrizione \\ \hline
token & json & Token per la sessione \\ \hline
host & string & Hostname del client  \\ \hline
port & int & Porta alla quale il client riceve le notifiche  \\ \hline
\end{tabular} \\
\subsubsection{Messaggi di notifica}
Un messaggio di notiica, inviato dal server al client, rispetta il formato:
\begin{lstlisting}[language=json]
{
  "msg_id": 22,
  "code": <notification code>,
  "text": <human readable meaning>
}
\end{lstlisting}
\begin{tabular}{|l | l | l |}
\hline
Campo & Tipo & Descrizione \\ \hline
code & int & codice di notifica \\
& & -1 Notifiche OK \\
& & 1 Asta chiusa\\
& & 2 Hai vinto\\ \hline
text & string & Messaggio di notifica \\ \hline
\end{tabular} \\


\subsection{Categorie}
I prodotti in vendita nel sistema di aste sono suddivisi per categorie merceologiche. E' possibile effettuare varie operazioni sulle cateogorie:
Ogni categoria è identificata in maniera univoca dal suo nome.
\subsubsection{Registrazione categoria}
Attraverso il seguente comando è possibile aggiungere una nuova categoria\\
Il nome della categoria sarà utilizzato per identificarla.

\begin{lstlisting}[language=json]
{
  "token": <token>,
  "msg_id": 31,
  "category": <category name>
}
\end{lstlisting}

\begin{tabular}{|l | l | l |}
\hline
Campo & Tipo & Descrizione \\ \hline
category & string & Nome della categoria, univoco.\\ \hline
\end{tabular} \\
\subsubsection{Lista delle categorie}
Attraverso il seguente comando è possibile ottenere una lista delle categorie
\begin{lstlisting}[language=json]
{
  "token": <token>,
  "msg_id": 33
}
\end{lstlisting}

Il formato della risposta è:
\begin{lstlisting}[language=json]
{
  "msg_id": -1,
  "response": <code>
  "categories":
  [
    <category1>,
    <category2>
  ]
}
\end{lstlisting}
\subsubsection{Ricerca di una categoria}
Attraverso il seguente comando è possibile ricercare tra le categorie per nome
\begin{lstlisting}[language=json]
{
  "token": <token>,
  "msg_id": 32,
  "category": <category name>
}
\end{lstlisting}

\begin{tabular}{|l | l | l |}
\hline
Campo & Tipo & Descrizione \\ \hline
category & string & Espressione da ricercare.\\ \hline
\end{tabular} \\

Il formato della risposta è lo stesso del precedente comando.
\subsection{Prodotti}
All'interno delle categorie merceologiche sono presenti i prodotti. E' possibile aggiungere nuovi prodotti, ricercare i prodotti per nome o visualizzarli tutti.
\subsubsection{Registra un nuovo prodotto}
Il seguente comando aggiunge una nuova asta per il determinato prodotto.
\begin{lstlisting}[language=json]
{
  "token": <token>,
  "msg_id": 41,
  "category": <cat_name>,
  "product": <prod_name>,
  "price": <price>
}
\end{lstlisting}

\begin{tabular}{|l | l | l |}
\hline
Campo & Tipo & Descrizione \\ \hline
category & string & categoria alla quale appartiene il prodotto. \\ \hline
product & string & nome del prodotto (univoco per categoria).\\ \hline
price & float & prezzo di partenza dell'asta.\\ \hline
\end{tabular} \\

\subsubsection{Ricerca di un prodotto}
Attraverso questo comando è possibile ricercare un prodotto all'interno di una categoria.
\begin{lstlisting}[language=json]
{
  "token": <token>,
  "msg_id": 42,
  "category": <category name>
  "product": <product name>
}
\end{lstlisting}

\begin{tabular}{|l | l | l |}
\hline
Campo & Tipo & Descrizione \\ \hline
category & string & Categoria alla quale appartiene il prodotto. \\ \hline
product & strin & Espressione da ricercare come nome del prodotto.\\ \hline
\end{tabular} \\
\subsubsection{Lista tutti i prodotti}
\begin{lstlisting}[language=json]
{
  "token": <token>,
  "msg_id": 43,
  "category": <category name>
}
\end{lstlisting}

\begin{tabular}{|l | l | l |}
\hline
Campo & Tipo & Descrizione \\ \hline
category & string & categoria  della quale si vuole la lista di prodotti. \\ \hline
\end{tabular} \\
\subsubsection{Messaggio di risposta}
La risposta del server eredita il formato base, introducendo un nuovo codice di risposta:\\
\begin{tabular}{|l | l | l |}
\hline
Campo & Tipo & Descrizione \\ \hline
response & int & '4' categoria non esistente \\ \hline
\end{tabular} \\


\subsection{Asta}
Una volta che i prodotti sono stati inseriti i vari utenti posso partecipare ad un asta offrendo un prezzo per l'oggeto.\\
L'asta termina quando si raggiunge il tempo limite, o quando il venditore decide di chiduerla. In entrambi i casi i partecipanti verranno notificati
E' anche possibile ritirare un offerta, ma solo nel caso sia l'offerta più alta.
\subsubsection{Offerta}
L'utente può effettuare un offerta per un prodotto già inserito.\\
L'offerta sarà valida solamente se non è doppia (l'offerta più alta al momento non è dell'utente) al fine di evitare che un utente possa far salire arbitrariamente il prezzo di un prodotto.\\
Nel caso l'offerta sia inferiore all'ultima offerta registrata allora riceveremo un errore.
\begin{lstlisting}[language=json]
{
  'token': s<token>,
  'msg_id': 51,
  'category': <category name>,
  'product': <product name>
  'price': <price>
}
\end{lstlisting}
\begin{tabular}{|l | l | l |}
\hline
Campo & Tipo & Descrizione \\ \hline
category & string & nome della categoria \\ \hline
product & string & nome del prodotto \\ \hline
price & float & valore dell'offerta \\ \hline
\end{tabular} \\

\subsubsection{Ritiro di un offerta}
E' possibile ritirare un offerta fatta ad un'asta, ma solo nel caso sia ancora l'offerta più alta.
\begin{lstlisting}[language=json]
{
  'token': s<token>,
  'msg_id': 52,
  'category': <category name>,
  'product': <product name>
}
\end{lstlisting}
\begin{tabular}{|l | l | l |}
\hline
Campo & Tipo & Descrizione \\ \hline
category & string & nome della categoria \\ \hline
product & string & nome del prodotto \\ \hline
\end{tabular} \\

\subsubsection{Chiusura di un'asta}
\begin{lstlisting}[language=json]
{
  'token': s<token>,
  'msg_id': 53,
  'category': <category name>,
  'product': <product name>
}
\end{lstlisting}
\begin{tabular}{|l | l | l |}
\hline
Campo & Tipo & Descrizione \\ \hline
category & string & nome della categoria \\ \hline
product & string & nome del prodotto \\ \hline
\end{tabular} \\
\subsubsection{Messaggio di risposta}
La risposta del server eredita il formato base, introducendo nuovi codici di risposta:\\
\begin{tabular}{|l | l | l |}
\hline
Campo & Tipo & Descrizione \\ \hline
response & int & '4' categoria non valida \\ \hline
response & int & '5' asta non valida\\ \hline
response & int &  '6' offerta non valida\\ \hline
\end{tabular} \\
\section{Implementazione}
\subsection{Librerie utilizzate}

\section{Tests}


\end{document}
