\section{Protocollo}
E' stato utlizzato JSON come formato per lo scambio di dati in quanto erano disponibili nella libreria standard python i metodi per serializzare e deserializzare gli oggetti.\\

Il formato di un generico  messaggio è:
\begin{lstlisting}[language=json]
{
  "msg_id": <id>
}
\end{lstlisting}
La prima cifra del campo msg\_id indica il tipo di richiesta, la seconda indica il metodo
\begin{itemize}
\item -1: Messaggio di risposta a un comando
\item 1: Autenticazione
\begin{itemize}
\item 11: Log in
\item 12: Log out
\end{itemize}
\item 2: Notifiche
\begin{itemize}
\item 21: Registra l'endpoint per le notifiche
\item 22: Messaggio di notifica
\end{itemize}
\item 3: Categorie
\begin{itemize}
\item 31: Registra nuova categoria
\item 32: Ricerca di una categoria
\item 33: Lista tutte le categorie
\end{itemize}
\item 4: Prodotti
\begin{itemize}
\item 41: Registra un nuovo prodotto
\item 42: Ricerca un prodotto
\item 43: Lista tutti i prodotti
\end{itemize}
\item 5: Asta
\begin{itemize}
\item 51: Fai un offerta
\item 52: Cancella la tua ultima offerta
\item 53: Chiudi l'asta

\end{itemize}

\end{itemize}

\subsection{Messaggio di risposta}
Il formato di un generico messaggio di risposta è il seguente.\\
Se non è esplicitamente indicato un'altro formato per la risposta,
 allora la risposta rispetterà quest'ultimo.
\begin{lstlisting}[language=json]
{
  "msg_id": -1
  "response" : <code>
}
\end{lstlisting}

\begin{tabular}{|l | l | l |}
\hline
Campo & Tipo & Descrizione \\ \hline
response & int & '1' in caso di successo  \\
& & '-1' comando non sia valido \\
& & '0' errore \\
& & '2' sessione scaduta \\
& & '3' sessione non valida \\ \hline
\end{tabular}\\
Questo messaggio può essere esteso dalle varie funzioni aggiungendo campi e codici di risposta.\\

\subsection{Autenticazione}
Il primo scambio di messaggi tra client e server è necessariamente di  autenticazione. Viene effettuato utilizzando un username ed una password.
Il server mantiene una lista di utenti con relative password, quando il client effettua l'autenticazione vengono confrontati e nel caso corrispondano l'autenticazione è avvenuta.
Per evitare di memorizzare le password in chiaro, è stato scelto di memorizzare e trasmettere l'hash md5 della password.
Il Client invia al server l'hash md5 della password inserita dall'utente, che lo confronta con quello memorizzato.

\subsubsection{Sessioni}
Al fine di gestire le sessioni  mantenendo un procollo stateless è stato scelto di generare un token di autenticazione ed inviarlo al client.
Una volta effettuato il login il server invia il token al client, questo lo utilizza per autenticare tutti i successivi comandi.\\
Al termine della scadenza il token perde di validità ed è necessario autenticarsi nuovamente.
Il token inviato al client è della forma:
\begin{lstlisting}[language=json]
{
  "username" : <user>
  "expiration": <expiration>
  "signature": <signed proof of autenticity>
}
\end{lstlisting}

\begin{tabular}{|l | l | l|}
\hline
Campo & Tipo & Descrizione \\ \hline
username & stringa & Username dell'utente \\ \hline
expiration & int & Data e ora di scadenza della sessione (in secondi) \\ \hline
signature & stringa & Codice di verifica del messaggio\\ \hline
\end{tabular}\\

Il campo signature è un hash dei precedenti due campi (username e expiration) più una chiave segreta generata dal server.
In questa maniera il server può verificare che il token sia autentico senza doverne mantenere una copia in memoria.
\subsubsection{Formato}
Il messaggio di autenticazione è del formato:
\begin{lstlisting}[language=json]
{
  "msg_id": 11,
  "user": <username>,
  "pass": <hash of pwd>
}
\end{lstlisting}

\begin{tabular}{|l | l | l |}
\hline
Campo & Tipo & Descrizione \\ \hline
user & stringa & Username dell'utente \\ \hline
pass & stringa & Hash MD5 della password \\ \hline
\end{tabular} \\
\vspace*{1em}

La risposta è del formato:
\begin{lstlisting}[language=json]
{
  "msg_id": -1,
  "response": <code>
  "token": <token>
}
\end{lstlisting}

\begin{tabular}{|l | l | l |}
\hline
Campo & Tipo & Descrizione \\ \hline
response & int & codice di risposta  \\ \hline
token & json & token per la sessione \\ \hline
\end{tabular} \\
\subsection{Gestione delle Notifiche}
Il server di aste può inviare delle notifiche asincrone ai client per comunicare aggiornamenti di stato.
Per esempio il client riceve una notifica quando un'asta viene terminata ed ogni volta che sua offerta viene superata.\\
\subsubsection{Registrazione}
Per ricevere delle notifiche asincrone il client  rimane in ascolto su una determinata porta, questa viene  comunicata insieme al proprio indirizzo  al server. Per fare questo è disponibile un comando. Di seguito il formato:\\
\begin{lstlisting}[language=json]
{
  "token": <token>,
  "msg_id": 21,
  "host": <hostname>,
  "port": <port>
}
\end{lstlisting}

\begin{tabular}{|l | l | l |}
\hline
Campo & Tipo & Descrizione \\ \hline
token & json & Token per la sessione \\ \hline
host & string & Hostname del client  \\ \hline
port & int & Porta alla quale il client riceve le notifiche  \\ \hline
\end{tabular} \\
\subsubsection{Messaggi di notifica}
Un messaggio di notiica, inviato dal server al client, rispetta il formato:
\begin{lstlisting}[language=json]
{
  "msg_id": 22,
  "code": <notification code>,
  "text": <human readable meaning>
}
\end{lstlisting}
\begin{tabular}{|l | l | l |}
\hline
Campo & Tipo & Descrizione \\ \hline
code & int & codice di notifica: \\
& & -1 Notifiche OK \\
& & 1 Asta chiusa\\
& & 2 Hai vinto\\
& & 3 Offerta superata\\ \hline

text & string & Messaggio di notifica \\ \hline
\end{tabular} \\


\subsection{Categorie}
I prodotti in vendita nel sistema di aste sono suddivisi per categorie merceologiche. E' possibile effettuare varie operazioni sulle categorie:
Ogni categoria è identificata in maniera univoca dal suo nome.
\subsubsection{Registrazione categoria}
Attraverso il seguente comando è possibile aggiungere una nuova categoria\\
Il nome della categoria è utilizzato per identificarla.

\begin{lstlisting}[language=json]
{
  "token": <token>,
  "msg_id": 31,
  "category": <category name>
}
\end{lstlisting}

\begin{tabular}{|l | l | l |}
\hline
Campo & Tipo & Descrizione \\ \hline
category & string & Nome della categoria, univoco.\\ \hline
\end{tabular} \\
\subsubsection{Lista delle categorie} \label{sssec:lc}
Attraverso il seguente comando è possibile ottenere una lista delle categorie
\begin{lstlisting}[language=json]
{
  "token": <token>,
  "msg_id": 33
}
\end{lstlisting}

Il formato della risposta è:
\begin{lstlisting}[language=json]
{
  "msg_id": -1,
  "response": <code>
  "categories":
  [
    <category1>,
    <category2>
  ]
}
\end{lstlisting}
\subsubsection{Ricerca di una categoria} \label{sssec:rc}
Attraverso il seguente comando è possibile ricercare tra le categorie per nome
\begin{lstlisting}[language=json]
{
  "token": <token>,
  "msg_id": 32,
  "category": <category name>
}
\end{lstlisting}

\begin{tabular}{|l | l | l |}
\hline
Campo & Tipo & Descrizione \\ \hline
category & string & Espressione da ricercare.\\ \hline
\end{tabular} \\

Il formato della risposta è lo stesso del precedente comando.
\subsection{Prodotti}
All'interno delle categorie merceologiche sono presenti i prodotti. E' possibile aggiungere nuovi prodotti, ricercare i prodotti per nome o visualizzarli tutti.
\subsubsection{Registra un nuovo prodotto}
Il seguente comando aggiunge una nuova asta per il determinato prodotto.
\begin{lstlisting}[language=json]
{
  "token": <token>,
  "msg_id": 41,
  "category": <cat_name>,
  "product": <prod_name>,
  "price": <price>
}
\end{lstlisting}

\begin{tabular}{|l | l | l |}
\hline
Campo & Tipo & Descrizione \\ \hline
category & string & categoria alla quale appartiene il prodotto. \\ \hline
product & string & nome del prodotto (univoco per categoria).\\ \hline
price & float & prezzo di partenza dell'asta.\\ \hline
\end{tabular} \\

\subsubsection{Ricerca di un prodotto} \label{sssec:rp}
Attraverso questo comando è possibile ricercare un prodotto all'interno di una categoria.
\begin{lstlisting}[language=json]
{
  "token": <token>,
  "msg_id": 42,
  "category": <category name>
  "product": <product name>
}
\end{lstlisting}

\begin{tabular}{|l | l | l |}
\hline
Campo & Tipo & Descrizione \\ \hline
category & string & Categoria alla quale appartiene il prodotto. \\ \hline
product & strin & Espressione da ricercare come nome del prodotto.\\ \hline
\end{tabular} \\
\subsubsection{Lista tutti i prodotti} \label{sssec:lp}
\begin{lstlisting}[language=json]
{
  "token": <token>,
  "msg_id": 43,
  "category": <category name>
}
\end{lstlisting}

\begin{tabular}{|l | l | l |}
\hline
Campo & Tipo & Descrizione \\ \hline
category & string & categoria  della quale si vuole la lista di prodotti. \\ \hline
\end{tabular} \\
\subsubsection{Messaggio di risposta}
La risposta del server eredita il formato base, introducendo un nuovo codice di risposta:\\
\begin{tabular}{|l | l | l |}
\hline
Campo & Tipo & Descrizione \\ \hline
response & int & '4' categoria non esistente \\ \hline
\end{tabular} \\


\subsection{Asta}
Una volta che i prodotti sono stati inseriti i vari utenti posso partecipare ad un asta offrendo un prezzo per l'oggeto.\\
L'asta termina quando si raggiunge il tempo limite, o quando il venditore decide di chiduerla. In entrambi i casi i partecipanti verranno notificati
E' anche possibile ritirare un offerta, ma solo nel caso sia l'offerta più alta.
\subsubsection{Offerta}
L'utente può effettuare un offerta per un prodotto già inserito.\\
Al fine di evitare che un utente possa far salire arbitrariamente il prezzo di un prodotto non è possibile inviare offerte successive da parte dello stesso utente..\\
Nel caso l'offerta sia inferiore all'ultima offerta registrata allora sarà restituito un errore.
\begin{lstlisting}[language=json]
{
  'token': <token>,
  'msg_id': 51,
  'category': <category name>,
  'product': <product name>
  'price': <price>
}
\end{lstlisting}
\begin{tabular}{|l | l | l |}
\hline
Campo & Tipo & Descrizione \\ \hline
category & string & nome della categoria \\ \hline
product & string & nome del prodotto \\ \hline
price & float & valore dell'offerta \\ \hline
\end{tabular} \\

\subsubsection{Ritiro di un offerta}
E' possibile ritirare un offerta fatta ad un'asta, ma solo nel caso sia ancora l'offerta più alta.
\begin{lstlisting}[language=json]
{
  'token': <token>,
  'msg_id': 52,
  'category': <category name>,
  'product': <product name>
}
\end{lstlisting}
\begin{tabular}{|l | l | l |}
\hline
Campo & Tipo & Descrizione \\ \hline
category & string & nome della categoria \\ \hline
product & string & nome del prodotto \\ \hline
\end{tabular} \\

\subsubsection{Chiusura di un'asta}
\begin{lstlisting}[language=json]
{
  'token': <token>,
  'msg_id': 53,
  'category': <category name>,
  'product': <product name>
}
\end{lstlisting}
\begin{tabular}{|l | l | l |}
\hline
Campo & Tipo & Descrizione \\ \hline
category & string & nome della categoria \\ \hline
product & string & nome del prodotto \\ \hline
\end{tabular} \\
\subsubsection{Messaggio di risposta}
La risposta del server eredita il formato base, introducendo nuovi codici di risposta:\\
\begin{tabular}{|l | l | l |}
\hline
Campo & Tipo & Descrizione \\ \hline
response & int & '4' categoria non valida \\ \hline
response & int & '5' asta non valida\\ \hline
response & int &  '6' offerta non valida\\ \hline
response & int &  '7' utente non autorizzato\\ \hline
response & int &  '8' asta chiusa senza vincitori\\ \hline
\end{tabular} \\