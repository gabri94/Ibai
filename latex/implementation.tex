\section{Implementazione}
Questo sisistema è stato implementato in Python 2.7, si divide in Client e Server, i quali sono composti da varie classi:
\subsection{Server}
Il server è stato implementato utilizzando più Thread. Il thread principale si occupa solamente di accettare le connessioni dagli utenti e smistarle. Per ogni connessione viene fatto partire un thread ''ClientManager'' che si occupa della comunicazione con il client.\\
E' stata scelta la porta 7652 per la comunicazione tra client e server. Per le notifiche la porta viene scelta casualmente dal sistema operativo.
\subsubsection{IbaiServer}
Questa è la classe principale del Server. Contiene le funzioni di inizializzazione del server e del database.
Contiene il loop principale all'interno del quale accetta le connessioni dai client e le passa alla classe ClientManager.
\ref{sssec:lc} \ref{sssec:rc} \ref{sssec:rp}  \ref{sssec:lp} 
\subsubsection{ClientManager}
Questa classe esegue la maggior parte delle operazioni relative ai client. Eredita la classe thread e definisce un metodo ''run'' che viene  eseguito all'avvio del thread.\\
Questo metodo è un loop di iterazione che riceve i comandi dal client. Una volta ricevuto un comando viene passato alla funzione “read\_command” che lo interpreta ed esegue la funzione appropriata. Le altre funzioni sono funzioni specifiche per ogni comando definito nel protocollo: sell, buy, bid, etc.\\
Sono stati implementati tutti i comandi definiti nel protocollo eccetto i comandi definiti nelle sezioni: 
\paragraph{\textbf{Sincronizzazione}}
I vari threads del server accedono contemporaneamente alla memoria condivisa del thread principale. Qui sono memorizzati gli utenti, le aste e le categorie. Per evitare problemi di accessi concorrenti alla memoria, tutte le funzioni che vi accedono sono decorate con una funzione detta ``Synchronizer'' che impedisce a due thread di eseguire un accesso contemporaneamente. Il blocco è effettuato per mezzo di un lock.

\subsubsection{model}
Modelli delle strutture dati utilizzate:
\paragraph{\textbf{Auction}}
Questa classe rappresenta un asta. Gli attributi di questo oggetto sono:
\begin{itemize}
\item{name}: Nome dell'oggetto in vendita (Case sensitive)
\item{price}: Prezzo di base dell'asta
\item{owner}: Utente che ha messo in vendita l'oggetto
\item{users}: Lista di utenti che hanno partecipato all'asta
\item{bids}: Lista di offerte con i relativi utenti
\end{itemize}
I metodi della classe sono:
\begin{itemize}
\item {bid(self, price, user)}: Inserisce un offerta dell'utente "user" e con prezzo "price"
\item {unbid(self, user)}: Rimuove l'ultima offerta dell'utente "user"
\item {winner(self)}: Ritorna l'utente che ha fatto l'ultima offerta più alta
\item {close(self, user)}: Se user corrisponde all'attributo "owner", chiude l'asta e notifica il vincitore.
\item {\_\_notify\_all(self, code, msg)}: Notifica tutti gli utenti iscritti all'asta con un messaggio
\end{itemize}
\paragraph{\textbf{Category}}
Questa classe rappresenta una categoria. I suoi attributi sono: 
\begin{itemize}
\item{name}: Nome della categoria merceologica
\item{auctions}: Lista di aste oggetti facenti parte della categoria
\end{itemize}
I suoi metodi sono:
\begin{itemize}
\item{add\_auction(self, auction)}: Aggiunge un oggetto alla categoria
\item{del\_auction(self, auction)}: Rimuove un oggetto dalla categoria
\item{search\_auction(self, name)}: Cerca un oggetto per nome nella categoria
\end{itemize}
\paragraph{\textbf{User}}
Questa classe rappresenta un utente, i suoi attributi sono:
\begin{itemize}
\item{name}: Nome utente
\item{password}: Hash MD5 della password dell'utente
\item{date}: Data di nascita dell'utente
\item{remote\_port}: Porta remota per l'invio delle notifiche
\item{remote\_host}: Host remoto per l'invio delle notifiche
\end{itemize}
I suoi metodi sono:
\begin{itemize}
\item{update\_notif\_socket(self, host, port)}: Aggiorna l'hostname e la porta dell'endpoint remoto per l'invio delle notifiche
\item{notify(self, code, msg)}: Notifica l'utente con un messaggio e un codice di notifica
\end{itemize}
\paragraph{Exceptions}
Qua sono definite tutte le eccezioni personalizzate usate nel programma.
\subsection{Client}
\subsubsection{IbaiClient}
Questa classe definisce un interfaccia client per collegarsi al server ed eseguire i comandi. E' stata implementata per semplificare lo svolgimento dei test.
Si divide in due parti:\\
La parte principale contiene tutti i metodi per interfacciarsi al server (register, sell, buy, unbuy, etc.) e i relativi metodi per interpretare la risposta del server.\\
La seconda parte invece è un thread che  sta in ascolto e riceve le notifiche. Quando viene chiamata la funzione "listen\_notify" dal thread principale, viene creato un secondo thread (questo), che sta in ascolto fino a che non termina il programma.
\subsection{Librerie utilizzate}
Per scrivere il programma sono state utilizzate alcune librerie di terze parti. Tutte sono presenti nel set di librerie standard di python:
\begin{itemize}
\item{\textbf{json}}: Questa libreria viene utilizzata per serializzare e de-serializzare i messaggi in formato json inviati sulla socket.
\item{\textbf{hmac}}: Questa liberia viene utilizzate per generare la firma presente nel token di autenticazione
\item{\textbf{socket}}: Questa libreria gestisce la comunicazione di rete
\item{\textbf{threading}}: Questa libreria gestisce i thread paralleli, sia lato server (gestione contemporanea di più client) che lato client(gestione delle notifiche asincrone)
\item{\textbf{time}}: Questa liberia viene usata per controllare la scadenza dei token di autenticazione
\item{\textbf{Queue}}: Questa libreria viene usata per la comunicazione tra i thread (lato client). In particolare è usata per memorizzare i messaggi di notifica non ancora processati.
\item{\textbf{haslib}}: Questa libreria viene utilizzata per generare l'hash MD5 della password utente
\item{\textbf{random}}: Questa libreria viene utilizzata per generare la chiave privata del server.
\item{\textbf{unittest}}: Questa libreria viene utilizzata per il testing dell'applicazione
\end{itemize}